% Basic Category Theory
% Tom Leinster <Tom.Leinster@ed.ac.uk>
% 
% Copyright (c) Tom Leinster 2014-2016
% 
% Introduction
% 

\chapter*{Introduction}
\label{ch:intro}

Category theory takes a bird's eye view of mathematics.  From high in the sky,
details become invisible, but we can spot patterns that were impossible to
detect from ground level.  How is the lowest common multiple of two numbers
like the direct sum of two vector spaces?  What do discrete topological
spaces, free groups, and fields of fractions have in common?  We will
discover answers to these and many similar questions, seeing patterns in
mathematics that you may never have seen before.

The most important concept in this book is that of \emph{universal%
%
\index{universal!property|(}
%
property}.  The further you go in mathematics, especially pure mathematics,
the more universal properties you will meet.  We will spend most of our
time studying different manifestations of this concept.

Like all branches of mathematics, category theory has its own special
vocabulary, which we will meet as we go along.  But since the idea of
universal property is so important, I will use this introduction to explain
it with no jargon at all, by means of examples.

Our first example of a universal property is very simple.

\begin{iexample} 
\label{eg:univ-terminal-set}
Let $1$%
%
\ntn{oneset}
%
denote a set with one%
%
\index{set!one-element}
%
element.  (It does not matter what this element is called.)  Then $1$ has
the following property:
% 
\begin{displaytext}
for all sets $X$, there exists a unique map from $X$ to $1$.
\end{displaytext}
% 
(In this context, the words `map', `mapping' and `function' all mean the
same thing.)

Indeed, let $X$ be a set.  There \emph{exists} a map $X \to 1$, because we
can define $f\from X \to 1$ by taking $f(x)$ to be the single element of
$1$ for each $x \in X$.  This is the \emph{unique} map $X \to 1$, because
there is no choice in the matter: any map $X \to 1$ must send each element
of $X$ to the single element of $1$.
\end{iexample}

Phrases of the form `there exists a unique%
%
\index{uniqueness}
% 
such-and-such satisfying some condition' are common in category theory.
The phrase means that there is one and only one such-and-such satisfying
the condition.  To prove the existence part, we have to show that there is
at least one.  To prove the uniqueness part, we have to show that there is
at most one; in other words, any two such-and-suches satisfying the
condition are equal.

Properties such as this are called `universal' because they state how the
object being described (in this case, the set $1$) relates to the entire
universe in which it lives (in this case, the universe of sets).  The
property begins with the words `\emph{for all} sets $X$', and therefore
says something about the relationship between $1$ and \emph{every} set $X$:
namely, that there is a unique map from $X$ to $1$.

\begin{iexample}        
\label{eg:univ-Z}
This example involves rings,%
%
\index{ring}
%
which in this book are always taken to have a multiplicative identity,
called $1$.  Similarly, homomorphisms of rings are understood to preserve
multiplicative identities.

The ring $\integers$%
%
\index{Z@$\integers$ (integers)!ring@as ring}
%
has the following property: for all rings $R$, there exists a unique
homomorphism $\integers \to R$.

To prove existence, let $R$ be a ring.  Define a function $\phi \from
\integers \to R$ by
\[
\phi(n)
=
\begin{cases}
\underbrace{1 + \cdots + 1}_n   &\text{ if } n > 0,     \\
0                               &\text{ if } n = 0,     \\
-\phi(-n)                       &\text{ if } n < 0
\end{cases}
\]
($n \in \integers$).  A series of elementary checks confirms that $\phi$ is
a homomorphism.

To prove uniqueness, let $R$ be a ring and let $\psi\from \integers \to R$
be a homomorphism.  We show that $\psi$ is equal to the homomorphism $\phi$
just defined.  Since homomorphisms preserve multiplicative identities,
$\psi(1) = 1$.  Since homomorphisms preserve addition,
\[
\psi(n) 
= 
\psi(\underbrace{1 + \cdots + 1}_n)
=
\underbrace{\psi(1) + \cdots + \psi(1)}_n
=
\underbrace{1 + \cdots + 1}_n
=
\phi(n)
\]
for all $n > 0$.  Since homomorphisms preserve zero, $\psi(0) = 0 = \phi(0)$.
Finally, since homomorphisms preserve negatives, $\psi(n) = -\psi(-n) =
-\phi(-n) = \phi(n)$ whenever $n < 0$.
\end{iexample}

Crucially, there can be essentially only \emph{one} object satisfying a
given universal property.  The word `essentially' means that two objects
satisfying the same universal property need not literally be equal, but
they are always isomorphic.  For example: 

\begin{ilemma}  
\label{lemma:Z-unique}
%
\index{universal!property!determines object uniquely}
%
Let $A$ be a ring with the following property: for all rings $R$, there exists
a unique homomorphism $A \to R$.  Then $A \iso \integers$.
\end{ilemma}

\begin{pf}
Let us call a ring with this property `initial'.  We are given that $A$
is initial, and we proved in Example~\ref{eg:univ-Z} that $\integers$ is
initial.  

Since $A$ is initial, there is a unique homomorphism $\phi\from A \to
\integers$.  Since $\integers$ is initial, there is a unique homomorphism
$\phi'\from \integers \to A$.  Now $\phi' \of \phi \from A \to A$ is a
homomorphism, but so too is the identity map $1_A\from A \to A$; hence,
since $A$ is initial, $\phi' \of \phi = 1_A$.  (This follows from the
uniqueness part of initiality, taking `$R$' to be $A$.)  Similarly, $\phi
\of \phi' = 1_\integers$.  So $\phi$ and $\phi'$ are mutually inverse, and
therefore define an isomorphism between $A$ and $\integers$.
\end{pf}

This proof has very little to do with rings.  It really belongs at a higher
level of generality.  To properly understand this, and to convey more fully
the idea of universal property, it will help to consider some more
complex examples.

\begin{iexample} 
\label{eg:univ-basis}
Let $V$ be a vector%
%
\index{vector space}
%
space with a basis $(v_s)_{s \in S}$.  (For example, if $V$ is
finite-dimensional then we might take $S = \{1, \ldots, n\}$.)  If $W$ is
another vector space, we can specify a linear map from $V$ to $W$ simply by
saying where the basis elements go.  Thus, for any $W$, there is a natural
one-to-one correspondence between
% 
\begin{displaytext}
linear maps $V \to W$
\end{displaytext}
% 
and
% 
\begin{displaytext}
functions $S \to W$.
\end{displaytext}
% 
This is because any function defined on the basis elements extends
uniquely to a linear map on $V$.

Let us rephrase this last statement.  Define a function $i\from S \to V$ by
$i(s) = v_s$ ($s \in S$).  Then $V$ together with $i$ has the following
universal property:
\[ 
\xymatrix{
S \ar[r]^i \ar[dr]_{\forall \text{ functions } f}  &
V \ar@{.>}[d]^{\exists ! \text{ linear } \bar{f}} \\
&
\forall W.
}
\]
This diagram means that for all vector spaces $W$ and all functions $f\from
S \to W$, there exists a unique linear map $\bar{f}: V \to W$ such that
$\bar{f} \of i = f$.  The symbol $\forall$%
%
\ntn{forall}
%
means `for all', and the symbols $\exists !$% 
% 
\ntn{ei}
%
 mean `there exists a unique'.%
%
\index{uniqueness}
%

Another way to say `$\bar{f} \of i = f$' is `$\bar{f}(v_s) = f(s)$ for all
$s \in S$'.  So, the diagram asserts that every function $f$ defined on the
basis elements extends uniquely to a linear map $\bar{f}$ defined on the
whole of $V$.  In other words still, the function
\[
\begin{array}{ccc}
\{ \text{linear maps } V \to W \}     &
\ \to \             &
\{ \text{functions } S \to W \}       \\
\bar{f}        &\mapsto        &\bar{f} \of i       
\end{array}
\]
is bijective. 
\end{iexample}

\begin{iexample} 
\label{eg:univ-discrete}
Given a set $S$, we can build a topological space $D(S)$%
%
\ntn{discrete-space}
%
by equipping $S$ with the \demph{discrete%
%
\index{topological space!discrete} 
%
topology}: all subsets are open.  With this topology, \emph{any} map from
$S$ to a space $X$ is continuous.

Again, let us rephrase this.  Define a function $i\from S \to D(S)$ by
$i(s) = s$ ($s \in S$).  Then $D(S)$ together with $i$ has the following
universal property:
\[
\xymatrix{
S \ar[r]^-i \ar[dr]_{\forall \text{ functions } f}       &
D(S) \ar@{.>}[d]^{\exists! \text{ continuous } \bar{f}}   \\
&
\forall X.
}
\]
In other words, for all topological spaces $X$ and all functions $f\from S \to
X$, there exists a unique continuous map $\bar{f}\from D(S) \to X$ such that
$\bar{f} \of i = f$.  The continuous map $\bar{f}$ is the same thing as the
function $f$, except that we are regarding it as a continuous map between
topological spaces rather than a mere function between sets.

You may feel that this universal property is almost too trivial to mean
anything.  But if we change the definition of $D(S)$~-- say from the
discrete to the indiscrete topology, in which the only open sets are
$\emptyset$ and $S$~-- then the property becomes false.  So this property
really does say something about the discrete topology.  What it says is
that all maps out of a discrete space are continuous.

Indeed, given $S$, the universal property determines $D(S)$ and $i$
uniquely (or rather, uniquely up to isomorphism; but who could want more?).
The proof of this is similar to that of Lemma~\ref{lemma:Z-unique} above
and Lemma~\ref{lemma:tensor-unique} below.
\end{iexample}

\begin{iexample} 
\label{eg:univ-tensor}
Given vector%
%
\index{vector space}
%
spaces $U$, $V$ and $W$, a \demph{bilinear%
%
\index{map!bilinear}
%
map} $f\from U \times V \to W$ is a function $f$ that is linear in each
variable:
% 
%\begin{align*}
%f(u, v_1 + \lambda v_2)         &=      
%f(u, v_1) + \lambda f(u, v_2),  \\
%f(u_1 + \lambda u_2, v)         &=      
%f(u_1, v) + \lambda f(u_2, v)   
%\end{align*}
% 
for all $u, u_1, u_2 \in U$, $v, v_1, v_2 \in V$, and scalars $\lambda$.  A
good example is the scalar product (dot product), which is a bilinear map
\[
\begin{array}{ccc}
\reals^n \times \reals^n        &\to            &\reals                 \\
(\mathbf{u}, \mathbf{v})        &\mapsto        &\mathbf{u}.\mathbf{v}
\end{array}
\]
of real vector spaces.  The vector product (cross product) $\reals^3 \times
\reals^3 \to \reals^3$ is also bilinear.  

Let $U$ and $V$ be vector spaces.  It is a fact that there is a `universal
bilinear map out of $U \times V$'.  In other words, there exist a certain
vector space $T$%
%
\index{tensor product|(}
%
and a certain bilinear map $b\from U \times V \to T$ with the following
universal property:
% 
\begin{equation}        
\label{eq:univ-tensor} 
\begin{array}{c}
\xymatrix{
U \times V \ar[r]^-b \ar[dr]_{\forall \text{ bilinear } f} &
T \ar@{.>}[d]^{\exists! \text{ linear } \bar{f}}     \\
&
\forall W.
}
\end{array}
\end{equation}
% 
Roughly speaking, this property says that bilinear maps out of $U \times V$
correspond one-to-one with linear maps out of $T$.

Even without knowing that such a $T$ and $b$ exist, we can immediately
prove that this universal property determines $T$ and $b$ uniquely up to
isomorphism.  The proof is essentially the same as that of
Lemma~\ref{lemma:Z-unique}, but looks more complicated because of the more
complicated universal property.
\end{iexample}

\begin{ilemma}   
\label{lemma:tensor-unique}
%
\index{universal!property!determines object uniquely}
%
Let $U$ and $V$ be vector spaces.  Suppose that $b\from U \times V \to T$
and $b'\from U \times V \to T'$ are both universal bilinear maps out of $U
\times V$.  Then $T \iso T'$.  More precisely, there exists a unique
isomorphism $j\from T \to T'$ such that $j \of b = b'$.
\end{ilemma}

In the proof that follows, it does not actually matter what `bilinear',
`linear' or even `vector space' mean.  The hard part is getting the logic
straight.  That done, you should be able to see that there is really only
one possible proof.  For instance, to use the universality of $b$, we will
have to choose some bilinear map $f$ out of $U\times V$.  There are only
two in sight, $b$ and $b'$, and we use each in the appropriate place.

\begin{pf}
In diagram~\eqref{eq:univ-tensor}, take $\Bigl(U \times V \toby{f} W\Bigr)$
to be $\Bigl(U \times V \toby{b'} T'\Bigr)$.  This gives a linear map
$j\from T \to T'$ satisfying $j \of b = b'$.  Similarly, using the
universality of $b'$, we obtain a linear map $j': T' \to T$ satisfying $j'
\of b' = b$:
\[
\xymatrix{
        &T \ar[d]^-j     \\
U \times V \ar[ur]^-b \ar[r]|-{b'} \ar[dr]_b      &
T' \ar[d]^{j'}  \\
&
T.
}
\]
Now $j' \of j\from T \to T$ is a linear map satisfying $(j' \of j) \of b =
b$; but also, the identity map $1_T\from T \to T$ is linear and satisfies
$1_T \of b = b$.  So by the uniqueness part of the universal property of
$b$, we have $j' \of j = 1_T$.  (Here we took the `$f$'
of~\eqref{eq:univ-tensor} to be $b$.)  Similarly, $j \of j' = 1_{T'}$.  So
$j$ is an isomorphism.  
\end{pf}

In Example~\ref{eg:univ-tensor}, it was stated that given vector spaces $U$
and $V$, there exists a pair $(T, b)$ with the universal property
of~\eqref{eq:univ-tensor}.  We just proved that there is essentially only
one such pair $(T, b)$.  The vector space $T$ is called the \demph{tensor
product} of $U$ and $V$, and is written as $U \otimes V$.%
%
\ntn{tensor}
%
Tensor products are very important in algebra.  They reduce the study of
bilinear maps to the study of linear maps, since a bilinear map out of $U
\times V$ is really the same thing as a linear map out of $U \otimes V$.

However, tensor products will not be important in this book.  The real
lesson for us is that it is safe to speak of \emph{the} tensor product, not
just \emph{a} tensor product, and the reason for that is
Lemma~\ref{lemma:tensor-unique}.  This is a general point that applies to
anything satisfying a universal property.

Once you know a universal property of an object, it often does no harm to
forget how it was constructed. For instance, if you look through a pile of
algebra books, you will find several different ways of constructing the
tensor product of two vector spaces.  But once you have proved that the
tensor product satisfies the universal property, you can forget the
construction.%
%
\index{tensor product|)} 
%
The universal property tells you all you need to know, because it
determines the object uniquely up to isomorphism.

\begin{iexample} 
\label{eg:univ-kernel}
Let $\theta\from G \to H$ be a homomorphism of groups.%
%
\index{group}
%
Associated with $\theta$ is a diagram
% 
\begin{equation}        
\label{eq:eq-groups}
%
\index{kernel}
%
\begin{array}{c}
\xymatrix@M+.5ex{
\ker(\theta) \ar@{^{(}->}[r]^-{\iota} &
G \ar@<.5ex>[r]^\theta \ar@<-.5ex>[r]_\epsln      &
H,
}
\end{array}
\end{equation}
% 
where $\iota$ is the inclusion of $\ker(\theta)$ into $G$ and $\epsln$ is
the trivial homomorphism.  `Inclusion'%
%
\index{inclusion}
%
means that $\iota(x) = x$ for all $x \in \ker(\theta)$, and `trivial' means
that $\epsln(g) = 1$ for all $g \in G$.  The symbol $\incl$%
%
\ntn{incl}
%
is often used for inclusions; it is a combination of a subset symbol
$\subset$ and an arrow.

The map $\iota$ into $G$ satisfies $\theta \of \iota = \epsln \of \iota$,
and is universal as such.  Exercise~\ref{ex:ker-groups} asks you to make
this precise.
\end{iexample}

Here is our final example of a universal property.

\begin{iexample} 
\label{eg:univ-van-Kampen}
Take a topological%
%
\index{topological space}
%
space covered by two open subsets: $X = U \cup V$.  The diagram
\[
\xymatrix@M+0.5ex{
U \cap V \ar@{^{(}->}[r]^-{i} \ar@{^{(}->}[d]_-{j}      &
U \ar@{^{(}->}[d]^-{j'} \\
V \ar@{^{(}->}[r]_-{i'} &
X
}
\]
of inclusion maps has a universal property in the world of topological spaces
and continuous maps, as follows:
% 
\begin{equation} 
\label{eq:pushout-vK}
\begin{array}{c}
\xymatrix@M+.5ex{
U \cap V \ar@{^{(}->}[r]^-{i} \ar@{^{(}->}[d]_-{j}      &
U \ar@{^{(}->}[d]^-{j'} \ar@/^/[ddr]^{\forall f} &
\\
V \ar@{^{(}->}[r]_-{i'} \ar@/_/[drr]_{\forall g} &
X \ar@{.>}[dr]|{\exists! h}     &
\\
&
&
\forall Y.
}
\end{array}
\end{equation}
% 
The diagram means that given $Y$, $f$ and $g$ such that $f \of i = g \of
j$, there is exactly one continuous map $h\from X \to Y$ such that $h \of
j' = f$ and $h \of i' = g$.

Under favourable conditions, the induced diagram 
\[
\xymatrix@M+0.5ex{
\pi_1(U \cap V) \ar[r]^-{i_*} \ar[d]_-{j_*}      &
\pi_1(U) \ar[d]^-{j'_*} \\
\pi_1(V) \ar[r]_-{i'_*} &
\pi_1(X)
}
\]
of fundamental%
%
\index{group!fundamental}
%
groups has the same property in the world of groups and group
homomorphisms.  This is \emph{van%
%
\index{van Kampen's theorem}
%
Kampen's theorem}.  In fact, van Kampen stated his theorem in a much more
complicated way.  Stating it transparently requires some categorical
language, but he was working in the 1930s, before category theory had been
born.
\end{iexample}

You have now seen several examples of universal properties.  As this book
progresses, we will develop different ways of talking about them.  Once we
have set up the basic vocabulary of categories and functors, we will study
\emph{adjoint functors}, then \emph{representable functors}, then
\emph{limits}.  Each of these provides an approach to universal properties,
and each places the idea in a different light.  For instance,
Examples~\ref{eg:univ-basis} and~\ref{eg:univ-discrete} can most readily be
described in terms of adjoint functors, Example~\ref{eg:univ-tensor} via
representable functors, and Examples~\ref{eg:univ-terminal-set},
\ref{eg:univ-Z}, \ref{eg:univ-kernel} and~\ref{eg:univ-van-Kampen} in terms
of limits.%
%
\index{universal!property|)}


\exs


\begin{iquestion}
Let $S$ be a set.  The \demph{indiscrete}%
%
\index{topological space!indiscrete}%
\index{indiscrete space} 
%
topological space $I(S)$%
%
\ntn{indiscrete-space}
%
is the space whose set of points is $S$ and whose only open subsets are
$\emptyset$ and $S$ itself.  Imitating Example~\ref{eg:univ-discrete}, find
a universal property satisfied by the space $I(S)$.
\end{iquestion}


\begin{iquestion}        
\label{ex:ker-groups}
Fix a group homomorphism $\theta\from G \to H$.  Find a universal property
satisfied by the pair $(\ker(\theta), \iota)$%
%
\index{kernel}
%
of diagram~\eqref{eq:eq-groups}.  (This property can~-- indeed, must~-- make
reference to $\theta$.)
\end{iquestion}


\begin{iquestion}
Verify the universal property shown in diagram~\eqref{eq:pushout-vK}.
\end{iquestion}


\begin{iquestion}       
\label{ex:Zx}
Denote by $\integers[x]$%
%
\ntn{poly-one}
%
the polynomial%
%
\index{ring!polynomial}
%
ring over $\integers$ in one variable.
% 
\begin{enumerate}[(b)]
\item   
\label{part:Zx-main}
Prove that for all rings $R$ and all $r \in R$, there exists a unique ring
homomorphism $\phi\from \integers[x] \to R$ such that $\phi(x) = r$.

\item 
Let $A$ be a ring and $a \in A$.  Suppose that for all rings $R$ and all
$r \in R$, there exists a unique ring homomorphism $\phi\from A \to R$ such that
$\phi(a) = r$.  Prove that there is a unique isomorphism $\iota\from
\integers[x] \to A$ such that $\iota(x) = a$.  
\end{enumerate}
\end{iquestion}


\begin{iquestion}
Let $X$ and $Y$ be vector spaces.  
% 
\begin{enumerate}[(b)]
\item   
\label{part:vs-prod}
For the purposes of this exercise only, a \emph{cone} is a triple $(V, f_1,
f_2)$ consisting of a vector space $V$, a linear map $f_1\from V \to X$, and
a linear map $f_2\from V \to Y$.  Find a cone $(P, p_1,
p_2)$ with the following property: for all cones $(V, f_1, f_2)$, there exists
a unique linear map $f\from V \to P$ such that $p_1 \of f = f_1$ and $p_2 \of f
= f_2$.

\item 
Prove that there is essentially only one cone with the property stated
in~\bref{part:vs-prod}.  That is, prove that if $(P, p_1, p_2)$ and $(P',
p'_1, p'_2)$ both have this property then there is an isomorphism $i\from P
\to P'$ such that $p'_1 \of i = p_1$ and $p'_2 \of i = p_2$.

\item   
\label{part:vs-coprod} 
For the purposes of this exercise only, a \emph{cocone} is a triple $(V,
f_1, f_2)$ consisting of a vector space $V$, a linear map $f_1\from X \to
V$, and a linear map $f_2\from Y \to V$.  Find a cocone $(Q, q_1, q_2)$
with the following property: for all cocones $(V, f_1, f_2)$, there exists
a unique linear map $f\from Q \to V$ such that $f \of q_1 = f_1$ and $f \of
q_2 = f_2$.

\item 
Prove that there is essentially only one cocone with the property stated
 in~\bref{part:vs-coprod}, in a sense that you should make precise.  
\end{enumerate}
\end{iquestion}

